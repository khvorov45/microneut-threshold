\documentclass[11pt]{article}

\usepackage[utf8]{inputenc}
\usepackage[T1]{fontenc}
\usepackage{lmodern}
\usepackage[table]{xcolor}
\usepackage{booktabs}
\usepackage{amsmath}
\usepackage[margin=1in]{geometry}
\usepackage{graphicx}
\usepackage[numbers]{natbib}
\usepackage{url}
\usepackage[bookmarks,bookmarksnumbered,bookmarksopen,hidelinks]{hyperref}
\usepackage{multirow}

\renewcommand{\familydefault}{\sfdefault}

\setlength{\parindent}{0em}
\setlength{\parskip}{1em}

\bibliographystyle{vancouver}

\title{Microneut threshold}

\author{Arseniy Khvorov}

\begin{document}

\maketitle

\section{Methods}

For every threshold,
the sensitivity estimate
was the proportion of PCR-positive subjects with titres
over the threshold (or equal to the threshold).
The specificity estimate
was the proportion of not PCR-positive subjects
with titres under the threshold.

In both cases the confidence interval was calculated using the Clopper-Pearson
method as implemented in the PropCIs package \cite{PropCIs} in R \cite{R}.

This way of doing it avoids having to assume how the titres are distributed.

To investigate the impact of symptom duration, I took the new data and
created subsets of it by excluding subjects based on symptom duration.
The idea is that
unless someone had symptoms for at least a certain number of days then
the antibody test is not meaningful and their results should be excluded.

\section{Results}

The results for the previous dataset provided by Suellen
(Figure~\ref{fig:suellen}), the new dataset (Figure~\ref{fig:kanta}) and
the combined dataset (Figure~\ref{fig:combined})
are in Table~\ref{tab:result-all}.

The results for different symptom duration thresholds are in
Table~\ref{tab:result-dur}. Only sensitivity
is affected because symptom duration only applies to those infected in
the given data. Sensitivity generally increases at all thresholds with
increasing minimum duration.

Going from no accounting for symptom duration to excluding subjects whose
symptoms lasted for less than 2 days before serum collection
(i.e. setting minimum symptom duration to 2) appears to
increase test sensitivity by 5-10\% across the titre thresholds.

Setting minimum symptom duration to 12 leads to no false negatives in the data
at any threshold from 20 to 35.

\input{../simple-test/result-all.tex}

\input{../simple-test/result-dur.tex}

\begin{figure}[htp]
    \centering
    \includegraphics[width=.8\textwidth]{../data-plot/suellen-hist.pdf}
    \caption{
        The old Suellen data.
    }
    \label{fig:suellen}
\end{figure}

\begin{figure}[htp]
    \centering
    \includegraphics[width=.8\textwidth]{../data-plot/kanta-hist.pdf}
    \caption{
        The new data.
    }
    \label{fig:kanta}
\end{figure}

\begin{figure}[htp]
    \centering
    \includegraphics[width=.8\textwidth]{../data-plot/all-real.pdf}
    \caption{
        The combined dataset.
    }
    \label{fig:combined}
\end{figure}

\bibliography{references}

All code used is available from
\url{https://github.com/khvorov45/microneut-threshold}

\end{document}
