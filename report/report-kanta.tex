\documentclass[11pt]{article}

\usepackage[utf8]{inputenc}
\usepackage[T1]{fontenc}
\usepackage{lmodern}
\usepackage[table]{xcolor}
\usepackage{booktabs}
\usepackage{amsmath}
\usepackage[margin=1in]{geometry}
\usepackage{graphicx}
\usepackage[numbers]{natbib}
\usepackage{url}
\usepackage[bookmarks,bookmarksnumbered,bookmarksopen,hidelinks]{hyperref}
\usepackage{multirow}

\renewcommand{\familydefault}{\sfdefault}

\setlength{\parindent}{0em}
\setlength{\parskip}{1em}

\bibliographystyle{vancouver}

\title{Microneut threshold}

\author{Arseniy Khvorov}

\begin{document}

\maketitle

\section{Methods}

For every threshold,
the sensitivity estimate
was the proportion of PCR-positive subjects with titres
over the threshold (or equal to the threshold).
The specificity estimate
was the proportion of not PCR-positive subjects
with titres under the threshold.

In both cases the confidence interval was calculated using the Clopper-Pearson
method as implemented in the PropCIs package \cite{PropCIs} in R \cite{R}.

This way of doing it avoids having to assume how the titres are distributed.

\section{Results}

The results for the previous dataset provided by Suellen
(Figure~\ref{fig:suellen}), the new dataset (Figure~\ref{fig:kanta}) and
the combined dataset (Figure~\ref{fig:combined})
are in Table~\ref{tab:result-all}.

\input{../simple-test/result-all.tex}

\begin{figure}[htp]
    \centering
    \includegraphics[width=.8\textwidth]{../data-plot/suellen-hist.pdf}
    \caption{
        The old Suellen data.
    }
    \label{fig:suellen}
\end{figure}

\begin{figure}[htp]
    \centering
    \includegraphics[width=.8\textwidth]{../data-plot/kanta-hist.pdf}
    \caption{
        The new data.
    }
    \label{fig:kanta}
\end{figure}

\begin{figure}[htp]
    \centering
    \includegraphics[width=.8\textwidth]{../data-plot/all-real.pdf}
    \caption{
        The combined dataset.
    }
    \label{fig:combined}
\end{figure}

\bibliography{references}

All code used is available from
\url{https://github.com/khvorov45/microneut-threshold}

\end{document}
