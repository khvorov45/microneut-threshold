\documentclass[]{article}
\usepackage{lmodern}
\usepackage{amssymb,amsmath}
\usepackage{ifxetex,ifluatex}
\usepackage{fixltx2e} % provides \textsubscript
\ifnum 0\ifxetex 1\fi\ifluatex 1\fi=0 % if pdftex
  \usepackage[T1]{fontenc}
  \usepackage[utf8]{inputenc}
\else % if luatex or xelatex
  \ifxetex
    \usepackage{mathspec}
  \else
    \usepackage{fontspec}
  \fi
  \defaultfontfeatures{Ligatures=TeX,Scale=MatchLowercase}
    \setmainfont[]{LiberationSans}
    \setsansfont[]{LiberationSans}
    \setmonofont[Mapping=tex-ansi]{LiberationMono}
\fi
% use upquote if available, for straight quotes in verbatim environments
\IfFileExists{upquote.sty}{\usepackage{upquote}}{}
% use microtype if available
\IfFileExists{microtype.sty}{%
\usepackage[]{microtype}
\UseMicrotypeSet[protrusion]{basicmath} % disable protrusion for tt fonts
}{}
\PassOptionsToPackage{hyphens}{url} % url is loaded by hyperref
\usepackage[unicode=true]{hyperref}
\hypersetup{
            pdftitle={Microneut Threshold},
            pdfauthor={Arseniy Khvorov},
            pdfborder={0 0 0},
            breaklinks=true}
\urlstyle{same}  % don't use monospace font for urls
\usepackage[margin=1in]{geometry}
\usepackage{longtable,booktabs}
% Fix footnotes in tables (requires footnote package)
\IfFileExists{footnote.sty}{\usepackage{footnote}\makesavenoteenv{long table}}{}
\usepackage{graphicx,grffile}
\makeatletter
\def\maxwidth{\ifdim\Gin@nat@width>\linewidth\linewidth\else\Gin@nat@width\fi}
\def\maxheight{\ifdim\Gin@nat@height>\textheight\textheight\else\Gin@nat@height\fi}
\makeatother
% Scale images if necessary, so that they will not overflow the page
% margins by default, and it is still possible to overwrite the defaults
% using explicit options in \includegraphics[width, height, ...]{}
\setkeys{Gin}{width=\maxwidth,height=\maxheight,keepaspectratio}
\IfFileExists{parskip.sty}{%
\usepackage{parskip}
}{% else
\setlength{\parindent}{0pt}
\setlength{\parskip}{6pt plus 2pt minus 1pt}
}
\setlength{\emergencystretch}{3em}  % prevent overfull lines
\providecommand{\tightlist}{%
  \setlength{\itemsep}{0pt}\setlength{\parskip}{0pt}}
\setcounter{secnumdepth}{5}
% Redefines (sub)paragraphs to behave more like sections
\ifx\paragraph\undefined\else
\let\oldparagraph\paragraph
\renewcommand{\paragraph}[1]{\oldparagraph{#1}\mbox{}}
\fi
\ifx\subparagraph\undefined\else
\let\oldsubparagraph\subparagraph
\renewcommand{\subparagraph}[1]{\oldsubparagraph{#1}\mbox{}}
\fi

% set default figure placement to htbp
\makeatletter
\def\fps@figure{htbp}
\makeatother


\title{Microneut Threshold}
\author{Arseniy Khvorov}
\date{16 April, 2020}

\begin{document}
\maketitle

\section{Overview}\label{overview}

The goal of this analysis is to find a diagnostic threshold titre
against COVID-19 of the microneutralisation assay.

This is predicated on the idea that infected people have higher titres
than non-infected people. An illustration is in Figure
\ref{fig:sim-hist-cont}.






\begin{figure}

{\centering \includegraphics{../data-plot/sim-hist-cont} 

}

\caption{Simulated titre data. Most non-infected people have
undetectable titres (the solid line, simulated from \(N(1, 1)\) on the
log-scale). Most infected people have detectable titres centered around
80 (the dashed line, simulated from \(N(4, 1)\) on the log-scale).}\label{fig:sim-hist-cont}
\end{figure}

Looking at Figure \ref{fig:sim-hist-cont}, we can draw a vertical line
at any titre value to act as a threshold. Anyone to the left of the line
would be considered not infected, anyone to the right --- infected. The
proportion of the titre distribution for the uninfected (the solid line)
to the left of the line would be the specificity of the test and the
proportion of the titre distribution for the infected (the dashed line)
to the right of the line would be the sensitivity of the test.

\section{Model}\label{model}

I considered a linear model for the log-titres

\begin{gather*}
T \sim N(\mu, \sigma) \\
\mu = \beta_0 + \beta_1 I
\end{gather*}

Where \(T\) is the log-titre and \(I\) is the infection status indicator
(1 --- infected, 0 --- not infected).

Undetectable titres (below 20) were given a value of 10. I treated this
value as if it were an actual measurement of 10.

\section{Model fitting}\label{model-fitting}

I fit the linear model using least squares. The procedure to obtain an
estimate of the sensitivity and specificity of different thresholds was
as follows

\begin{enumerate}
\def\labelenumi{\arabic{enumi}.}
\item
  Obtain the estimated log-titre mean \(m\) and standard error \(s\) of
  that mean from the fit for both the infected and the uninfected
  groups. Obtain estimated residual error \(r\).
\item
  For each group, generate a random number \(m_t\) from \(N(m, s^2)\).
  This is one possible true mean of the titre distribution for the
  group. Hence the distribution \(N(m_t, r^2)\) is one possible true
  titre distribution for the group.
\item
  For the uninfected group, calculate \(P(X < t)\) where \(t\) is the
  threshold log-titre and \(X \sim N(m_t, r^2)\). This is the
  specificity. For the infected group, calculate \(P(X > t)\). This is
  the sensitivity.
\item
  Repeat steps 2-3 many times (I did 4,000) for different titre
  thresholds (I chose 25 equally spaces thresholds between 20 and 80).
  This will yield samples for sensitivity and specificity for each titre
  threshold. Quantiles of these samples can act as bounds for interval
  estimates of sensitivity and specificity.
\end{enumerate}

\section{Results}\label{results}

Interval estimates for sensitivity and specificity of threshold-based
tests are presented in Figure \ref{fig:testchar}.





\begin{figure}

{\centering \includegraphics{../testchar-plot/suellen-testchar} 

}

\caption{Estimated sensitivity (top panel) and specificity (bottom
panel) of a threshold titre test. Bounds of the shaded region are the
95\% confidence interval. The solid line is the median estimate.}\label{fig:testchar}
\end{figure}

\end{document}
