\documentclass[]{article}
\usepackage{lmodern}
\usepackage{amssymb,amsmath}
\usepackage{ifxetex,ifluatex}
\usepackage{fixltx2e} % provides \textsubscript
\ifnum 0\ifxetex 1\fi\ifluatex 1\fi=0 % if pdftex
  \usepackage[T1]{fontenc}
  \usepackage[utf8]{inputenc}
\else % if luatex or xelatex
  \ifxetex
    \usepackage{mathspec}
  \else
    \usepackage{fontspec}
  \fi
  \defaultfontfeatures{Ligatures=TeX,Scale=MatchLowercase}
    \setmainfont[]{LiberationSans}
    \setsansfont[]{LiberationSans}
    \setmonofont[Mapping=tex-ansi]{LiberationMono}
\fi
% use upquote if available, for straight quotes in verbatim environments
\IfFileExists{upquote.sty}{\usepackage{upquote}}{}
% use microtype if available
\IfFileExists{microtype.sty}{%
\usepackage[]{microtype}
\UseMicrotypeSet[protrusion]{basicmath} % disable protrusion for tt fonts
}{}
\PassOptionsToPackage{hyphens}{url} % url is loaded by hyperref
\usepackage[unicode=true]{hyperref}
\hypersetup{
            pdftitle={Microneut Threshold},
            pdfauthor={Arseniy Khvorov},
            pdfborder={0 0 0},
            breaklinks=true}
\urlstyle{same}  % don't use monospace font for urls
\usepackage[margin=1in]{geometry}
\usepackage{longtable,booktabs}
% Fix footnotes in tables (requires footnote package)
\IfFileExists{footnote.sty}{\usepackage{footnote}\makesavenoteenv{long table}}{}
\usepackage{graphicx,grffile}
\makeatletter
\def\maxwidth{\ifdim\Gin@nat@width>\linewidth\linewidth\else\Gin@nat@width\fi}
\def\maxheight{\ifdim\Gin@nat@height>\textheight\textheight\else\Gin@nat@height\fi}
\makeatother
% Scale images if necessary, so that they will not overflow the page
% margins by default, and it is still possible to overwrite the defaults
% using explicit options in \includegraphics[width, height, ...]{}
\setkeys{Gin}{width=\maxwidth,height=\maxheight,keepaspectratio}
\IfFileExists{parskip.sty}{%
\usepackage{parskip}
}{% else
\setlength{\parindent}{0pt}
\setlength{\parskip}{6pt plus 2pt minus 1pt}
}
\setlength{\emergencystretch}{3em}  % prevent overfull lines
\providecommand{\tightlist}{%
  \setlength{\itemsep}{0pt}\setlength{\parskip}{0pt}}
\setcounter{secnumdepth}{5}
% Redefines (sub)paragraphs to behave more like sections
\ifx\paragraph\undefined\else
\let\oldparagraph\paragraph
\renewcommand{\paragraph}[1]{\oldparagraph{#1}\mbox{}}
\fi
\ifx\subparagraph\undefined\else
\let\oldsubparagraph\subparagraph
\renewcommand{\subparagraph}[1]{\oldsubparagraph{#1}\mbox{}}
\fi

% set default figure placement to htbp
\makeatletter
\def\fps@figure{htbp}
\makeatother


\title{Microneut Threshold}
\author{Arseniy Khvorov}
\date{16 April, 2020}

\begin{document}
\maketitle

\section{Overview}\label{overview}

The goal of this analysis is to find a diagnostic threshold titre
against COVID-19 of the microneutralisation assay.

This is predicated on the idea that infected people have higher titres
than non-infected people. An illustration is in Figure
\ref{fig:sim-hist-cont}.





\begin{figure}

{\centering \includegraphics{../data-plot/sim-hist-cont} 

}

\caption{Simulated titre data. Most non-infected people have
undetectable titres (the solid line). Most infected people have
detectable titres centered around 80 (the dashed line).}\label{fig:sim-hist-cont}
\end{figure}

Looking at Figure \ref{fig:sim-hist-cont}, we can draw a vertical line
at any titre value to act as a threshold. Anyone to the left of the line
would be considered not infected, anyone to the right --- infected. The
proportion of the titre distribution for the uninfected (the solid line)
to the left of the line would be the specificity of the test and the
proportion of the titre distribution for the infected (the dashed line)
to the right of the line would be the sensitivity of the test.

\section{Model}\label{model}

I considered a linear model for the log-titres

\begin{gather*}
T \sim N(\mu, \sigma) \\
\mu = \beta_0 + \beta_1 I
\end{gather*}

Where \(T\) is the log-titre and \(I\) is the infection status indicator
(1 --- infected, 0 --- not infected).

\end{document}
